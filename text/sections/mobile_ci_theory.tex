\section{Mobile CI/CD}

Coming now from the "classical" CI/CD approach, it is not possible to instantly apply the known model to the mobile world, it needs some adoption. Before we do so, we have to consider the following points:

\begin{itemize}
	
	\item \textbf{Mobile application have a high UI focus} \\
	UI tests are always more expensive to create than for "standard" code - and this is even more valid for mobile applications. Additionally from the known challenges from desktop applications like clickable fields, listener states, dependencies between views \& co, a mobile device introduces complexity
	\\
	
	\item \textbf{Mobile applications have per default a build tool} \\
	Both iOS (XCode) and Android (AndroidStudio \& gradle) come along with their own build system and IDE, making the question if the application is build manually or automated with a tool pointless.
	\\
	
	\item \textbf{There is no standard production environment} \\
	Due to the nature of the mobile device world, there is no standard device towards a deployment can happen. Especially but not only with Android there is a huge variety of screen sizes and operating system versions. Therefore automated testing has to be a compromise between covered variations and invested efforts. 
	\\
	
	\item \textbf{Emulation is expensive} \\
	The consequence of the previous point is that the testing necessarily is connected with emulating the OS for each test - due to the high share of required UI tests, a majority of testing can not be done independent from the mobile OS. But emulating is expensive in terms of time and resources.
	
\end{itemize}

\subsection{Adopted CI/CD model}
As consequence to the mentioned points, we will adopt the practise checklist of Fowler by completely removing 2) (inherited to the build tools) and changing 7) to "Test in a Emulated Environment".

\subsection{Google Best Practices \& tools}